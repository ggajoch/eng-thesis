\chapter{Introduction}
    Ionizing radiation is one of the major concerns during space mission development, both manned and unmanned. Just as the human body is affected by radiation, so are electronic components, which will fail under certain conditions.

    A special design methodology, radiation hardening, must be implemented for all satellite components, which dramatically increases mission costs. Some assumptions are made during satellite planning, such as radiation tolerances, above which the mission can fail. Absorbed dose can be predicted by simulations, but unusual events like solar flares can alter predictions to unacceptable levels. For monitoring the absorbed dose, most satellites have on-board Total Ionizing Dose (TID) sensors, allowing to deorbit or move satellite to graveyard orbit before it fails.

    Usually, these sensors are very expensive and it is hard to cut the cost due to custom ASIC design. They are also large and require a lot of power to operate. However, recent publications suggest that Commercial Off-The-Shelf (COTS) transistors can be used to assemble an absorbed dose sensor.

    To date, very few small student-satellites (e.g. CubeSats) have TID sensors on-board. This is mainly due to their cost, but also to limited time, space and power resources. At present, this sensor is not critical in Low Earth Orbit (LEO), when, following failure, a satellite will deorbit by means of atmospheric drag. In the near future, as a consequence of expanding CubeSat market beyond LEO, the possibility of satellite collisions is expected to grow significantly. This will force CubeSats to start implementing more radiation-hardening techniques, which, on pico- and nano-satellites, mainly consists of COTS components radiation tests and real-time operation monitoring. This opens the need for CubeSat TID sensors, which are currently not available on the market.

    In this thesis, the design of an absorbed dose sensor is presented. The thesis aims at presenting design requirements and solutions, along with simulations and preliminary tests. The presented sensor is planned to be flown on-board PW-Sat2 student satellite, in Q4 2017.

    Brief description of thesis chapters:
    \begin{itemize}
        \item Abbreviations, conventions - present abbreviations and conventions used within this thesis,
        \item Introduction - this chapter, description of the thesis aims,
        \item Principles - introduces reader to radiation related problems and explains the theory of operation,
        \item Requirements - presents design requirements for this particular sensor, because it is designed specifically for PW-Sat2 satellite,
        \item Sensor design - presents high-level sensor design phase, explaining its operation on block and system level,
        \item Engineering model - describes the sensor model developed during this thesis, its design and simulations,
        \item Tests - presents results of conducted sensor tests,
        \item Future work - briefly describes the required next steps until flight solution is ready,
        \item Summary - summarizes thesis, work and outcome.
    \end{itemize}
