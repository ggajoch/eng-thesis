\chapter{Introduction}
    Ionizing radiation is one of major concerns during space missions development, both manned and unmanned. As well as human body is affected by radiation, so are electronic components, which will fail in certain conditions.

    Special design methodology, radiation hardening, have to be implemented for every component in satellites, which dramatically increases costs of the mission. Some assumptions are made during satellite planning, such as radiation tolerance, above which mission can fail. Dose can be predicted by simulations, but unusual events like solar flares can alter predictions to unacceptable levels. For monitoring absorbed dose, most satellites have on-board Total Ionizing Dose (TID) sensors.

    Usually, this sensors are very expensive and it is hard to cut the cost due to custom ASIC design. They are also large and require a lot of power to operate. But, recent publications suggests that Commercial Off-The-Shelf (COTS) transistors can be used as a absorbed dose sensor.

    Up to date, very few small student-satellites (e.g. CubeSats) have TID sensors on-board. This is mainly to their cost, but also due to limited time, space and power resources. This sensor is not critical in Low Earth Orbit (LEO), when satellite after fail will deorbit by atmospheric drag. But, CubeSats are expanding beyond this point, when satellite fail can cause a lot of problems, due to possible collisions with other satellites, and very long decay time. This forces CubeSats to start implementing more radiation-hardening techniques, which main of them on pico- and nano-satellites is COTS components radiation tests and real-time monitoring. This opens need for CubeSat TID sensors, which are currently not available on the market.

    In this thesis, design of absorbed dose sensor is presented. Thesis aims at presenting design requirements and solutions, along with simulations and preliminary tests. This sensor is planned to be flown on-board PW-Sat2 student satellite, in Q4 2017.

    Brief description of thesis chapters:
    \begin{itemize}
        \item Abbreviations, conventions - present abbreviations and conventions used among this thesis,
        \item Introduction - this chapter, description of the aim of the thesis,
        \item Principles - introduces reader to radiation related problems and explains theory of operation,
        \item Requirements - presents design requirement for this particular sensor, because it is designed specifically for PW-Sat2 satellite,
        \item Sensor design - present high-level sensor design phase, explaining its operation on block and system level,
        \item Engineering model - describes sensor model developed during this thesis, its design and simulations,
        \item Tests - presents results of conducted sensor tests,
        \item Future work - briefly describes required next steps until flight solution is ready,
        \item Summary - summarizes thesis, work and outcome.
    \end{itemize}

    
