\chapter{Principles}

\section{Space radiation effects on electronics}
In space electronics design there are two major radiation effects to be considered. Each has different background and different mitigation schemes.

\subsection{Single Event Effects}
Single Event Effects are connected with very high energy particles striking electronic component. Is can cause different problems:


Major mitigation scheme is to 
\subsection{Total Ionising Dose}


\section{Need for TID radiation measurements}
    As mentioned before, TID can be very dangerous - ionising radiation accumulated can cause any device to stop functioning. Therefore, devices have to be tested for amount of radiation they can withstand ("radiation tests"). The radiation flux on orbit is well-described, but the TID accumulated during mission should be monitored to be able to abort and utilise sattelite before its fail. 

\section{PW-Sat2 mission}
    PW-Sat2 is Polish second student satellite, mainly build by students from Warsaw Univeristy of Technology. The cooperation between author and PW-Sat2 team led to this thesis.
    
\subsection{Main purpose}
    Main goal of PW-Sat2 is to test deorbitation sail mechanism. The sail (2x2 m size) will open from the back of the satellite and cause the orbit to decay faster.
    
    % photo 
\subsection{Orbit \& lifetime}
    PW-Sat2 mission is planned to last for 40 days. After 40 days sail will be opened and satellite will deorbit in couple of weeks time.
    
    PW-Sat2 will be launched on top of Falon9 rocket from SpaceX company, planned Q4 2017. The final orbit is 565 km sun-synchronous orbit.
    
    
\subsection{Radiation analysis}
    To estimate radiation dose accumulated by PW-Sat2 the simulation in SPENVIS envirnoment was performed. 
    % radiation patterns
    On orbit there are clearly seen Van-Allen radiation belts and South Atlantic Anomaly. 
    
    The total dose accumulated by satellite on this orbit is about $N rad$.