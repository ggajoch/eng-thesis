\chapter{Principles}

\section{Radiation effects on electronic devices}
    There are two basic effects of radiation on silicon components:
    \begin{itemize}
        \item ionizing effect
        \item displacement damage
    \end{itemize}

    Those two effect are responsible for changing parameters of semiconductor devices, which after some time can lead to failure of the device. Main source of this radiation are Gamma (ionization) and Neutron particles (displacement).

    In space electronics during analysis and design two major problems are considered - Single Event Effects (SEE) and Total Ionizing Dose (TID). Every silicon and silica device is susceptible to both of those - and both have to be considered during product design, development and testing.

    \subsection{Single Events Effects}
        Single Event Effects are connected with generation of electron-hole pair in semiconductor, when material is exposed to ionizing radiation. Amount of pairs generated is proportional to energy deposited. For semiconductor device parameter $LET_{th}$ (Linear Energy Transfer Threshold) is defined, being a measure of how susceptible the device is. For particles with Linear Energy Transfer (LET - normalized particle energy per mass of the absorbing material), below this threshold no effect will be observed.

        Single Event Effects are divided into two groups - non-destructive (fully recoverable, possibly after power cycle) and destructive (permanent damage) effects. Shortly those are described below, defined as in \cite{ECSS_Q_ST_60_15C}.

        \bigskip\textbf{Non-destructive effects}
        \begin{itemize}
            \item \textbf{Single Event Upset} - especially memory-based devices (like microprocessors, memories, Field Programmable Gate Array - FPGA) are vulnerable. This phenomenon will possibly alter the state of cell in memory - causing memory corruption. This can lead to complete failure of device if this is not corrected.

            \item \textbf{Single Event Functional Interrupt} - subset of SEU - this effect cause the system to latch in non-recoverable state (e.g. by switching to wrong state in state machine). Only option is to reset circuit to back to known state.

            \item \textbf{Single Event Transient} - are formed as a voltage/current spurious pulses generated by charge induced by striking particle. This can cause different problems - from disturbing analog electronics up to causing switch of digital circuit. This effect strongly depends on size of feature in silica.
        \end{itemize}

        \bigskip\textbf{Destructive effects}
        \begin{itemize}
            \item \textbf{Single Event Latch-up} - particle striking can cause turning on parasitic thyristor in CMOS structure. This will lead to effectively shorting voltage supply to ground, causing overheat and damage to the device.

            \item \textbf{Single Event Gate Rupture} - high energy particle comes through thin gate (especially in MOS transistors) can cause generation of electron-holes pairs in gate and substrate - causing high electric field across gate. When this effect is strong enough it can cause permanent damage to transistor.

            \item \textbf{Single Event Burnout} - ion that traverses the transistor structure (through the source) can induce a current flow that turns on the parasitic npn transistor. This leads to effective short circuit and damage to the device.
        \end{itemize}

        \bigskip\textbf{Mitigation techniques}

        Below recommended mitigation techniques for SEE were listed:
        \begin{itemize}
            \item SEU - redundancy, memory scrubbing,
            \item SEFI - watchdog, proper reset sequence,
            \item SET - use lower-integration scale devices, implement protection resistors etc.
            \item SEL - implement overcurrent circuits (like Latch-up Current Limiters),
            \item SEGR, SEB - use higher $LET_{th}$ devices
        \end{itemize}

    \subsection{Total Ionizing Dose}
        TID is defined as total energy absorbed during exposure. This can be caused by any kind of radiation, behaving differently in every semiconductor device. In general, TID successively degrades electronic device parameters in time, causing them to stop functioning when critical irradiation was reached. Effect in p-MOSFET transistor is described in section \ref{Radiation_effects_on_MOS_transistors}.

        Every qualified device for spacecraft system will have tested value of TID value, for example by MIL-STD-883G, Test Method 1019.7 \cite{MIL_STD_883}. For example, ADC128S102QML-SP have a guaranteed value of 100 kRad.

\section{Need for TID radiation dosimetry}
    During spacecraft mission accumulated radiation level should be monitored to not excess guaranteed values for components. For example, near end of its lifetime, spacecraft can be commanded to deorbit into atmosphere or move to parking orbit - before fail can occur, causing losing control of spacecraft.
    Absorbed dose simulation and estimation is possible, but results are averaged by long period of time. Constantly flying by South Atlantic Anomaly or Van Allen belts can cause radiation estimation to be inaccurate, so near all spacecrafts implement sensor which constantly monitor radiation level absorbed by its electronics.

\section{On-line TID radiation dosimetry}

\section{RadFET Theory}

    \subsection{Radiation effects on MOS transistors}
    \label{Radiation_effects_on_MOS_transistors}

    \subsection{Threshold voltage measurement}

    \subsection{Temperature dependencies}
