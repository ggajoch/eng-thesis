% \documentclass[pdflatex,11pt]{aghdpl}
% \documentclass{aghdpl}               % przy kompilacji programem latex
\documentclass[pdflatex,en]{aghdpl}  % praca w języku angielskim
\usepackage[polish]{babel}
\usepackage[utf8]{inputenc}

% dodatkowe pakiety
\usepackage{enumerate}
\usepackage{listings}
\lstloadlanguages{TeX}

\lstset{
  literate={ą}{{\k{a}}}1
           {ć}{{\'c}}1
           {ę}{{\k{e}}}1
           {ó}{{\'o}}1
           {ń}{{\'n}}1
           {ł}{{\l{}}}1
           {ś}{{\'s}}1
           {ź}{{\'z}}1
           {ż}{{\.z}}1
           {Ą}{{\k{A}}}1
           {Ć}{{\'C}}1
           {Ę}{{\k{E}}}1
           {Ó}{{\'O}}1
           {Ń}{{\'N}}1
           {Ł}{{\L{}}}1
           {Ś}{{\'S}}1
           {Ź}{{\'Z}}1
           {Ż}{{\.Z}}1
}

%---------------------------------------------------------------------------

\author{Marcin Szpyrka}
\shortauthor{M. Szpyrka}

\titlePL{Przygotowanie pracy dyplomowej w~systemie \LaTeX}
\titleEN{Thesis in \LaTeX}

\shorttitlePL{Przygotowanie pracy dyplomowej w~systemie \LaTeX} % skrócona wersja tytułu jeśli jest bardzo długi
\shorttitleEN{Thesis in \LaTeX}

\thesistypePL{Praca magisterska}
\thesistypeEN{Master of Science Thesis}

\supervisorPL{dr hab. Marcin Szpyrka}
\supervisorEN{Marcin Szpyrka Ph.D}

\date{2011}

\departmentPL{Katedra Automatyki}
\departmentEN{Department of Automatics}

\facultyPL{Wydział Elektrotechniki, Automatyki, Informatyki i Elektroniki}
\facultyEN{Faculty of Electrical Engineering, Automatics, Computer Science and Electronics}

\acknowledgements{Serdecznie dziękuję \dots tu ciąg dalszych podziękowań np. dla promotora, żony, sąsiada itp.}


%---------------------------------------------------------------------------

\begin{document}

\titlepages

\tableofcontents
\clearpage

\chapter{Wprowadzenie}
\label{cha:wprowadzenie}

\LaTeX~jest systemem składu umożliwiającym tworzenie dowolnego typu dokumentów (w~szczególności naukowych i technicznych) o wysokiej jakości typograficznej (\cite{Dil00}, \cite{Lam92}). Wysoka jakość składu jest niezależna od rozmiaru dokumentu -- zaczynając od krótkich listów do bardzo grubych książek. \LaTeX~automatyzuje wiele prac związanych ze składaniem dokumentów np.: referencje, cytowania, generowanie spisów (treśli, rysunków, symboli itp.) itd.

\LaTeX~jest zestawem instrukcji umożliwiających autorom skład i wydruk ich prac na najwyższym poziomie typograficznym. Do formatowania dokumentu \LaTeX~stosuje \TeX a (wymiawamy 'tech' -- greckie litery $\tau$, $\epsilon$, $\chi$). Korzystając z~systemu składu \LaTeX~mamy za zadanie przygotować jedynie tekst źródłowy, cały ciężar składania, formatowania dokumentu przejmuje na siebie system.

%---------------------------------------------------------------------------

\paragraph{X}
\subparagraph{Y}

\section{Cele pracy}
\label{sec:celePracy}


\subsection{A}
\subsubsection{B}

Celem poniższej pracy jest zapoznanie studentów z systemem \LaTeX~w zakresie umożliwiającym im samodzielne, profesjonalne złożenie pracy dyplomowej w systemie \LaTeX.


%---------------------------------------------------------------------------

\section{Zawartość pracy}
\label{sec:zawartoscPracy}

W rodziale~\ref{cha:pierwszyDokument} przedstawiono podstawowe informacje dotyczące struktury dokumentów w \LaTeX u. Alvis~\cite{Alvis2011} jest językiem 



















\chapter{Pierwszy dokument}
\label{cha:pierwszyDokument}

W rozdziale tym przedstawiono podstawowe informacje dotyczące struktury prostych plików \LaTeX a. Omówiono również metody kompilacji plików z zastosowaniem programów \emph{latex} oraz \emph{pdflatex}.

%---------------------------------------------------------------------------

\section{Struktura dokumentu}
\label{sec:strukturaDokumentu}

Plik \LaTeX owy jest plikiem tekstowym, który oprócz tekstu zawiera polecenia formatujące ten tekst (analogicznie do języka HTML). Plik składa się z dwóch części:
\begin{enumerate}%[1)]
\item Preambuły -- określającej klasę dokumentu oraz zawierającej m.in. polecenia dołączającej dodatkowe pakiety;

\item Części głównej -- zawierającej zasadniczą treść dokumentu.
\end{enumerate}


\begin{lstlisting}
\documentclass[a4paper,12pt]{article}      % preambuła
\usepackage[polish]{babel}
\usepackage[utf8]{inputenc}
\usepackage[T1]{fontenc}
\usepackage{times}

\begin{document}                           % część główna

\section{Sztuczne życie}

% treść
% ąśężźćńłóĘŚĄŻŹĆŃÓŁ

\end{document}
\end{lstlisting}

Nie ma żadnych przeciwskazań do tworzenia dokumentów w~\LaTeX u w~języku polskim. Plik źródłowy jest zwykłym plikiem tekstowym i~do jego przygotowania można użyć dowolnego edytora tekstów, a~polskie znaki wprowadzać używając prawego klawisza \texttt{Alt}. Jeżeli po kompilacji dokumentu polskie znaki nie są wyświetlane poprawnie, to na 95\% źle określono sposób kodowania znaków (należy zmienić opcje wykorzystywanych pakietów).


%---------------------------------------------------------------------------

\section{Kompilacja}
\label{sec:kompilacja}


Załóżmy, że przygotowany przez nas dokument zapisany jest w pliku \texttt{test.tex}. Kolejno wykonane poniższe polecenia (pod warunkiem, że w pierwszym przypadku nie wykryto błędów i kompilacja zakończyła się sukcesem) pozwalają uzyskać nasz dokument w formacie pdf:
\begin{lstlisting}
latex test.tex
dvips test.dvi -o test.ps
ps2pdf test.ps
\end{lstlisting}
%
lub za pomocą PDF\LaTeX:
\begin{lstlisting}
pdflatex test.tex
\end{lstlisting}

Przy pierwszej kompilacji po zmiane tekstu, dodaniu nowych etykiet itp., \LaTeX~tworzy sobie spis rozdziałów, obrazków, tabel itp., a dopiero przy następnej kompilacji korzysta z tych informacji.

W pierwszym przypadku rysunki powinny być przygotowane w~formacie eps, a~w~drugim w~formacie pdf. Ponadto, jeżeli używamy polecenia \texttt{pdflatex test.tex} można wstawiać grafikę bitową (np. w formacie jpg).



%---------------------------------------------------------------------------

\section{Narzędzia}
\label{sec:narzedzia}


Do przygotowania pliku źródłowego może zostać wykorzystany dowolny edytor tekstowy. Niektóre edytory, np. Emacs, mają wbudowane moduły ułatwiające składanie tekstów w LaTeXu (kolorowanie składni, skrypty kompilacji, itp.).

Jednym z bardziej znanych środowisk do składania dokumentów  \LaTeX a jest {\em Kile}. Aplikacja dostępna jest dla środowiska KDE począwszy od wersji 2. Zawiera edytor z podświetlaną składnią, zestawy poleceń \LaTeX a, zestawy symboli matematycznych, kreatory tabel, macierzy, skrypty kompilujące i konwertujące podpięte są do poleceń w menu aplikacji (i pasków narzędziowych), dostępne jest sprawdzanie pisowni, edytor obsługuje projekty (tzn. dokumenty składające się z~wielu plików), umożliwia przygotowanie i~zarządzanie bibliografią, itp.

Na stronie \underline{\texttt{http://kile.sourceforge.net/screenshots.php}} zamieszczono kilkanaście zrzutów ekranu środowiska {\em Kile}, które warto przejrzeć, by wstępnie zapoznać się z~możliwościami programu.

Bardzo dobrym środowiskiem jest również edytor gEdit z wtyczką obsługującą \LaTeX a. Jest to standardowy edytor środowiska Gnome. Po instalacji wtyczki obsługującej \LaTeX a, edytor nie ustępuje funkcjonalnościom środowisku Kile, a jest zdecydowanie szybszy w działaniu. Lista dostępnych wtyczek dla tego edytora znajduje się pod adresem \underline{\texttt{http://live.gnome.org/Gedit/Plugins}}. Inne polecane wtyczki to: 
\begin{itemize}
\item Edit shortcuts -- definiowanie własnych klawiszy skrótu;
\item Line Tools -- dodatkowe operacje na liniach tekstu;
\item Multi-edit -- możliwość jednoczesnej edycji w wielu miejscach tekstu;
\item Zoom -- zmiana wielkości czcionki edytora z użyciem rolki myszy;
\item Split View -- możliwość podziału okna edytora na 2 części. 
\end{itemize}



%---------------------------------------------------------------------------

\section{Przygotowanie dokumentu}
\label{sec:przygotowanieDokumentu}

Plik źródłowy \LaTeX a jest zwykłym plikiem tekstowym. Przygotowując plik
źródłowy warto wiedzieć o kilku szczegółach:

\begin{itemize}
\item
Poszczególne słowa oddzielamy spacjami, przy czym ilość spacji nie ma znaczenia.
Po kompilacji wielokrotne spacje i tak będą wyglądały jak pojedyncza spacja.
Aby uzyskać {\em twardą spację}, zamiast znaku spacji należy użyć znaku {\em
tyldy}.

\item
Znakiem końca akapitu jest pusta linia (ilość pusty linii nie ma znaczenia), a
nie znaki przejścia do nowej linii.

\item
\LaTeX~sam formatuje tekst. \textbf{Nie starajmy się go poprawiać}, chyba, że
naprawdę wiemy co robimy.
\end{itemize} 



\chapter{Principles}

\section{Space radiation effects on electronics}
In space electronics design there are two major radiation effects to be considered. Each has different background and different mitigation schemes.

\subsection{Single Event Effects}
Single Event Effects are connected with very high energy particles striking electronic component. Is can cause different problems:


Major mitigation scheme is to 
\subsection{Total Ionising Dose}


\section{Need for TID radiation measurements}
    As mentioned before, TID can be very dangerous - ionising radiation accumulated can cause any device to stop functioning. Therefore, devices have to be tested for amount of radiation they can withstand ("radiation tests"). The radiation flux on orbit is well-described, but the TID accumulated during mission should be monitored to be able to abort and utilise sattelite before its fail. 

\section{PW-Sat2 mission}
    PW-Sat2 is Polish second student satellite, mainly build by students from Warsaw Univeristy of Technology. The cooperation between author and PW-Sat2 team led to this thesis.
    
\subsection{Main purpose}
    Main goal of PW-Sat2 is to test deorbitation sail mechanism. The sail (2x2 m size) will open from the back of the satellite and cause the orbit to decay faster.
    
    % photo 
\subsection{Orbit \& lifetime}
    PW-Sat2 mission is planned to last for 40 days. After 40 days sail will be opened and satellite will deorbit in couple of weeks time.
    
    PW-Sat2 will be launched on top of Falon9 rocket from SpaceX company, planned Q4 2017. The final orbit is 565 km sun-synchronous orbit.
    
    
\subsection{Radiation analysis}
    To estimate radiation dose accumulated by PW-Sat2 the simulation in SPENVIS envirnoment was performed. 
    % radiation patterns
    On orbit there are clearly seen Van-Allen radiation belts and South Atlantic Anomaly. 
    
    The total dose accumulated by satellite on this orbit is about $N rad$.
\include{rozdzial4}

1. Introduction

2. Terms, definitions and abbreviated terms 
2.1. Abbreviated terms
2.2. Conventions

3. Principles
3.1. Space radiation effects on electronics
3.1.1. Single Events Effects
3.1.2. Total Ionising Dose
3.2. Need for TID radiation measurements




5. Design requirements
5.1. Sensor requirements
5.1.2. Required sensitivity
5.1.3. Required accuracy
5.2. Applicable standards

3.3. PW-Sat2 mission
3.3.1. Main purpose
3.3.2. Orbit \& lifetime
3.3.3. Radiation analysis

5.3. Electrical requirements
5.1.1. Electronics stack
5.1.2. On-Board Computer
5.1.3. Other sensors
5.1.4. Payload board
5.3.1. Power ICD
5.3.2. Data exchange ICD
5.3.3. Radiation immunity
5.3.4. Reliability of components

5.4. Mechanical requirements
5.4.1. PCB stack \& PCB restrictions
5.4.2. Space available
5.4.3. Vibration
5.4.4. Operation temperature
5.4.5. Thermal cycles





6. Sensor design
4.2. TID measurement technique
6.1.1. Accumulating particles in MOSFET
4.2.1. Physical phenomena background

6.2. Block diagram

6.1. Components selection
6.1.1. MOSFET
6.1.2. ADC
6.1.3. Microcontroller
6.1.4. Passives
6.2. Tradeoff analysis
6.2.1. Measurement channels
6.2.2. Measurement accuracy
6.2.3. Die temperature measurement
6.2.3.1. Build-in ESD diode
6.2.3.2. Body diode
6.2.4. Measurement period
6.2.5. Averaging
6.2.6. Bias of MOSFET gate

7. Sensor implementation
7.1. Analog front-end
7.1.1. Current source
7.1.2. Threshold voltage readout
7.1.3. Temperature measurement
7.2. Digital
7.2.1. Microcontroller
7.2.2. ADC
7.2.3. OBC interface
7.3. Software design
7.3.1. Other tasks running on the same processor
7.3.2. AVR-HAL
7.3.3. I2C-slave module
7.3.4. Measurement algorithm

8. Sensor prototypes
8.1. Calibration prototype v0.1
8.2. Calibration prototype v0.2
8.3. TID Sensor prototype board

9. Sensor production
9.1. Payload board
9.1.1. PCB layout 
9.1.2. 3D model
9.2. PCB production
9.3. Component soldering

10. Sensor tests
10.1. Measurement noise
10.2. Temperature stability
10.3. Operational temprature
10.4. Radiation tests
10.5. Estimated lifetime on different orbits
10.6. Digital interfaces

11. Summary

12. Bibliography

Figures
Tables



% itd.
% \appendix
% \include{dodatekA}
% \include{dodatekB}
% itd.

\bibliographystyle{alpha}
\bibliography{bibliografia}
%\begin{thebibliography}{1}
%
%\bibitem{Dil00}
%A.~Diller.
%\newblock {\em LaTeX wiersz po wierszu}.
%\newblock Wydawnictwo Helion, Gliwice, 2000.
%
%\bibitem{Lam92}
%L.~Lamport.
%\newblock {\em LaTeX system przygotowywania dokumentów}.
%\newblock Wydawnictwo Ariel, Krakow, 1992.
%
%\bibitem{Alvis2011}
%M.~Szpyrka.
%\newblock {\em {On Line Alvis Manual}}.
%\newblock AGH University of Science and Technology, 2011.cccccc
%\newblock \\\texttt{http://fm.ia.agh.edu.pl/alvis:manual}.
%
%\end{thebibliography}

\end{document}
