% \documentclass[pdflatex,11pt]{aghdpl}
% \documentclass{aghdpl}               % przy kompilacji programem latex
\documentclass[pdflatex,en]{aghdpl}  % praca w języku angielskim
\usepackage[polish]{babel}
\usepackage[utf8]{inputenc}

% dodatkowe pakiety
\usepackage{enumerate}
\usepackage{listings}
\lstloadlanguages{TeX}

\lstset{
  literate={ą}{{\k{a}}}1
           {ć}{{\'c}}1
           {ę}{{\k{e}}}1
           {ó}{{\'o}}1
           {ń}{{\'n}}1
           {ł}{{\l{}}}1
           {ś}{{\'s}}1
           {ź}{{\'z}}1
           {ż}{{\.z}}1
           {Ą}{{\k{A}}}1
           {Ć}{{\'C}}1
           {Ę}{{\k{E}}}1
           {Ó}{{\'O}}1
           {Ń}{{\'N}}1
           {Ł}{{\L{}}}1
           {Ś}{{\'S}}1
           {Ź}{{\'Z}}1
           {Ż}{{\.Z}}1
}

%---------------------------------------------------------------------------

\author{Grzegorz Gajoch}
\shortauthor{G. Gajoch}

\titlePL{Budowa sensora pochłoniętej dawki promieniowania jonizującego (TID) dla pico-satelitów typu CubeSat}
\titleEN{Total Ionizing Dose (TID) sensor for CubeSat~pico-satellites}

\shorttitlePL{Budowa sensora pochłoniętej dawki promieniowania jonizującego (TID) dla pico-satelitów typu CubeSat} % skrócona wersja tytułu jeśli jest bardzo długi
\shorttitleEN{Total Ionizing Dose (TID) sensor for CubeSat~pico-satellites}

\thesistypePL{Praca inżynierksa}
\thesistypeEN{Bachelor of Science Thesis}

\supervisorPL{dr inż. Cerazy Worek}
\supervisorEN{Cerazy Worek Ph.D}

\date{2016}

\departmentPL{Katedra Elektroniki}
\departmentEN{Department of Electronics}

\facultyPL{Wydział Informatyki, Elektroniki i Telekomunikacji}
\facultyEN{Faculty of Computer Science, Electronics and Telecommunications}

\acknowledgements{Serdecznie dziękuję \dots tu ciąg dalszych podziękowań np. dla promotora, żony, sąsiada itp.}


%---------------------------------------------------------------------------

\begin{document}

\titlepages

\tableofcontents
\clearpage

\chapter{Principles}

\section{Space radiation effects on electronics}
In space electronics design there are two major radiation effects to be considered. Each has different background and different mitigation schemes.

\subsection{Single Event Effects}
Single Event Effects are connected with very high energy particles striking electronic component. Is can cause different problems:


Major mitigation scheme is to 
\subsection{Total Ionising Dose}


\section{Need for TID radiation measurements}
    As mentioned before, TID can be very dangerous - ionising radiation accumulated can cause any device to stop functioning. Therefore, devices have to be tested for amount of radiation they can withstand ("radiation tests"). The radiation flux on orbit is well-described, but the TID accumulated during mission should be monitored to be able to abort and utilise sattelite before its fail. 

\section{PW-Sat2 mission}
    PW-Sat2 is Polish second student satellite, mainly build by students from Warsaw Univeristy of Technology. The cooperation between author and PW-Sat2 team led to this thesis.
    
\subsection{Main purpose}
    Main goal of PW-Sat2 is to test deorbitation sail mechanism. The sail (2x2 m size) will open from the back of the satellite and cause the orbit to decay faster.
    
    % photo 
\subsection{Orbit \& lifetime}
    PW-Sat2 mission is planned to last for 40 days. After 40 days sail will be opened and satellite will deorbit in couple of weeks time.
    
    PW-Sat2 will be launched on top of Falon9 rocket from SpaceX company, planned Q4 2017. The final orbit is 565 km sun-synchronous orbit.
    
    
\subsection{Radiation analysis}
    To estimate radiation dose accumulated by PW-Sat2 the simulation in SPENVIS envirnoment was performed. 
    % radiation patterns
    On orbit there are clearly seen Van-Allen radiation belts and South Atlantic Anomaly. 
    
    The total dose accumulated by satellite on this orbit is about $N rad$.

% 1. Introduction
% 
% 2. Terms, definitions and abbreviated terms 
% 2.1. Abbreviated terms
% 2.2. Conventions
% 
% 3. Principles
% 3.1. Space radiation effects on electronics
% 3.1.1. Single Events Effects
% 3.1.2. Total Ionising Dose
% 3.2. Need for TID radiation measurements
% 
% 
% 
% 
% 5. Design requirements
% 5.1. Sensor requirements
% 5.1.2. Required sensitivity
% 5.1.3. Required accuracy
% 5.2. Applicable standards
% 
% 3.3. PW-Sat2 mission
% 3.3.1. Main purpose
% 3.3.2. Orbit \& lifetime
% 3.3.3. Radiation analysis
% 
% 5.3. Electrical requirements
% 5.1.1. Electronics stack
% 5.1.2. On-Board Computer
% 5.1.3. Other sensors
% 5.1.4. Payload board
% 5.3.1. Power ICD
% 5.3.2. Data exchange ICD
% 5.3.3. Radiation immunity
% 5.3.4. Reliability of components
% 
% 5.4. Mechanical requirements
% 5.4.1. PCB stack \& PCB restrictions
% 5.4.2. Space available
% 5.4.3. Vibration
% 5.4.4. Operation temperature
% 5.4.5. Thermal cycles
% 
% 
% 
% 
% 
% 6. Sensor design
% 4.2. TID measurement technique
% 6.1.1. Accumulating particles in MOSFET
% 4.2.1. Physical phenomena background
% 
% 6.2. Block diagram
% 
% 6.1. Components selection
% 6.1.1. MOSFET
% 6.1.2. ADC
% 6.1.3. Microcontroller
% 6.1.4. Passives
% 6.2. Tradeoff analysis
% 6.2.1. Measurement channels
% 6.2.2. Measurement accuracy
% 6.2.3. Die temperature measurement
% 6.2.3.1. Build-in ESD diode
% 6.2.3.2. Body diode
% 6.2.4. Measurement period
% 6.2.5. Averaging
% 6.2.6. Bias of MOSFET gate
% 
% 7. Sensor implementation
% 7.1. Analog front-end
% 7.1.1. Current source
% 7.1.2. Threshold voltage readout
% 7.1.3. Temperature measurement
% 7.2. Digital
% 7.2.1. Microcontroller
% 7.2.2. ADC
% 7.2.3. OBC interface
% 7.3. Software design
% 7.3.1. Other tasks running on the same processor
% 7.3.2. AVR-HAL
% 7.3.3. I2C-slave module
% 7.3.4. Measurement algorithm
% 
% 8. Sensor prototypes
% 8.1. Calibration prototype v0.1
% 8.2. Calibration prototype v0.2
% 8.3. TID Sensor prototype board
% 
% 9. Sensor production
% 9.1. Payload board
% 9.1.1. PCB layout 
% 9.1.2. 3D model
% 9.2. PCB production
% 9.3. Component soldering
% 
% 10. Sensor tests
% 10.1. Measurement noise
% 10.2. Temperature stability
% 10.3. Operational temprature
% 10.4. Radiation tests
% 10.5. Estimated lifetime on different orbits
% 10.6. Digital interfaces
% 
% 11. Summary
% 
% 12. Bibliography
% 
% Figures
% Tables



% itd.
% \appendix
% \include{dodatekA}
% \include{dodatekB}
% itd.

\bibliographystyle{alpha}
\bibliography{bibliografia}

\end{document}
